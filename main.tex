\documentclass[12pt]{article}
\usepackage{my_preamble}

% Document Starts
\begin{document}

\maketitle 

\begin{abstract}
\end{abstract}

\tableofcontents

\pagebreak
% <++>


\section{Proposition}

An atomic prop cannot be broek down into smaller propositions.

A compound proposition is composed of atomics props.

\TB{Atomic}
\begin{itemize}
  \item I am graduating.
  \item I am applying for grad school.
\end{itemize}

\TB{Compound}
\begin{itemize}
  \item I am not graduating
  \item I am graduating implies im applying for grad school
\end{itemize}

\section{Logical Arguments}

An \TB{argument} is a set of props, consiting of zero or more premises.

\bboxex
\TB{Premises:} If I am applying for grad schools, then I must be graduating.
I am graduating.
\ebox
\bboxex
\TB{Conclusion:} I am applying for grad school.
\ebox

If the concl doesn't follow from prem then the argument is invalid.

% lecture 2


\section{Propositional Logic}
\subsection{Notations and Lp}
\newcommand{\Lp}[1][]{\TM{\mathcal{L}^p_{#1}}}

\bbox
\begin{nota}[Symbols]\label{nota:sym}
  \(\,\)\begin{itemize}
    \item \TB{Proposition Symbols.} Used for atomic formulas. We'll use lowercase 
      letters, \(\{a,b,c,\dots\}\).
    \item \TB{Connections.} \(\neg,\wedge,\vee,\to,\leftrightarrow\).
    \item \TB{Parems.} Denotes order.
  \end{itemize}
  Let Lp be the language of propositional logic.
  \[\wedge\wedge\vee\is\TM{not\;legal}\]
  \[(p\is\TM{not\;legal}\]
  We defined a tokenizer. We'll now define the \TB{parser}.
\end{nota}
\ebox


\bbox
\begin{defn}[Well Formed Expressions (WFE)]\label{defn:well_exp}
  \(\,\)
  \begin{enumerate}
    \item a propositional symbol is a well formed expression.
      \[p\is\TM{WFE}\]
    \item If \(A\in\mathrm{Form}\TM{Lp}\implies(\neg A)\in\mathrm{Form}\TM{Lp}\).
    \item If \(A,B\in\mathrm{Form}\Lp\implies (A\wedge B)\in\mathrm{Form}\Lp\).
    \item If \(A,B\in\mathrm{Form}\Lp\implies (A\vee B)\in\mathrm{Form}\Lp\).
    \item If \(A,B\in\mathrm{Form}\Lp\implies (A\to B)\in\mathrm{Form}\Lp\).
    \item If \(A,B\in\mathrm{Form}\Lp\implies (A\leftrightarrow B)\in\mathrm{Form}\Lp\).
  \end{enumerate}
\end{defn}
\ebox
\begin{verbatim}
data WFE t = PropSym t |
             ExprBin (t -> t -> t) (WFE t) (WFE t) |
             ExprUn (t -> t) (WFE t)

eval :: WFE t -> t
eval wfe = case wfe of
  PropSym t -> t
  ExprBim fn l r -> fn (eval l) (eval r)
  ExprUn fn u -> fn (eval u)


neg :: Bool -> Bool
neg pred = if pred then False else True

conj :: Bool -> Bool -> Bool
conj a b = if a then b else False

inj :: Bool -> Bool -> Bool
inj a b = (neg a) `conj` (neg b)
     -- = (conj `on` neg) a b

xor :: Bool -> Bool -> Bool
xor a b = (a `ing` b) `conj` ((neg . conj) a b)
\end{verbatim}
To make inline Lp work, we need to establish operator prior and associativity.
\bboxnote
\begin{nota}[Operator Priority]\label{nota:oper_prior} Operator prior is as follows:
\begin{enumerate}
  \item \(\neg\)
  \item \(\wedge\is\TM{left\;assoc}\)
  \item \(\vee\is\TM{left\;assoc}\)
  \item \(\implies\is\TM{right\;assoc}\)
  \item \(\iff\is\TM{left\;assoc}\)
\end{enumerate}
\end{nota}
\ebox
\bboxex
\[((\neg p)\vee q)=\neg p\vee q\]
\[(p\wedge q)\vee(r\wedge p)=p\wedge q\vee r\wedge q\]
\[p\to q\to r=p\to(q\to r)\]
\[p\wedge q\wedge r=(p\wedge q)\wedge r\]
\ebox

\section{Translating English into Prop Logic}

\subsection{Examples}
\bboxex
\begin{equation*}
  \begin{split}
    s&:=\T{I am applying to grad schools}\\
    j&:=\T{I am applying to jobs}\\
    g&:=\T{I am graduating}\\
    \T{s or j}&=s\vee j\\
    \T{i am either S or J but not S and J}&=(s\vee j)\wedge\neg(s\wedge j)\\
                                          &=s\iff\neg j\\
                                          &=(s\implies\neg j)\wedge(\neg j\implies s)\\
    (a\vee b)\wedge\neg(a\wedge b)&:=a\oplus b\\
    \T{s if g}&=g\implies s\\
    \T{s only if g}&=s\implies g\\
    \T{storm}\implies\T{rain}&=\T{rain if storm}\\
                             &=\T{it's raining if it's storming}\\
                             &=\T{storm only if rain}\\
                             &=\T{it's storming only if it's raining}\\
    \T{g is sufficient for s}&=g\implies s\\
    \T{g is necessary for s}&=s\implies g\\
    \T{Although g, i am not j}&=g\wedge\neg j\\
    \oplus=\neg\circ\leftrightarrow
  \end{split}
\end{equation*}
\ebox


\bbox
\begin{lem}[Balanced Paranthesis]\label{lem:balance_parans_in_lp}
  Every formula in \(\mathrm{Form}\Lp\) has balanced paranths.
\end{lem}
\ebox
\bboxproof
\begin{proof}
  Let \(A\) be an arbitrary formula in \(\mathrm{Form}\Lp\). The following
  proof is by \TB{structural induction}. Let \(R(A)\) be the property
  that \(LP(A)=RP(A)\). Letting \(LP(A)\) be the number of Left paranthesis'
  in \(A\). Let \(RP(A)\) be the number of Right paranthesis' in \(A\).

  \TB{Base Case}: \(A\) is atomic, \(A=p\) for some prop p.
  \[LP(A)=RP(A)=0\]

  \TB{Inductive Case 1}: \(A=\neg B\) for some \(B\in\mathrm{Form}\Lp\).
  Our IH says \(LP(B)=RP(B)\).
  \[LP(A)=LP((\neg B))=1 + LP(B)=1+RP(B)=RP((\neg B))=RP(A)\]

  \TB{Inductive Case 2}: Let \((\diamond)\) be a generic binary operator
  \((\diamond):\Lp\times\Lp\to\Lp\). 
  \(A=(B\diamond C),\T{ for some }B,C\in\TR{Form}\Lp\), with
  \(LP(B)=RP(B)\) and \(LP(C)=RP(C)\) by IH.
  \[LP(A)=LP((B\diamond C))=1+LP(B)+LP(C)\]
  \[=1+RP(B)+RP(C)=RP((B\diamond C))=RP(A)\]

  So by the principal of structural induction, \(R(A)\) holds.
\end{proof}
\ebox

% lecture 3
% last time talking about syntax of prop logic.

thm: for any \(A \in \TR{Form}(LP),\;LP(A)=RP(A)\), proven above

\begin{verbatim}
# Machine Proof of Above in Roc
Inductive formula : Type :=
  | Atom : string  -> formula
  | Not  : formula -> formula
  | And  : formula -> formula -> formula
  | Or   : formula -> formula -> formula
  | Imp  : formula -> formula -> formula
  | Iff  : formula -> formula -> formula

Fixpoint lparans (f : formula) : nat :=
  mathc f with
  | Atom _ => 0
  | Not f1 => lparans f1 + 1
  | And f1 f2 => lparans f1 + lparns f2 + 1
  | And f1 f2 => lparans f1 + lparns f2 + 1
  | And f1 f2 => lparans f1 + lparns f2 + 1
  | And f1 f2 => lparans f1 + lparns f2 + 1


Fixpoint rparans (f : formula) : nat :=
  mathc f with
  | Atom _ => 0
  | Not f1 => rparans f1 + 1
  | And f1 f2 => rparans f1 + lparns f2 + 1
  | And f1 f2 => rparans f1 + lparns f2 + 1
  | And f1 f2 => rparans f1 + lparns f2 + 1
  | And f1 f2 => rparans f1 + lparns f2 + 1

Theorem lparans_eq_rparens : forall f : formula, lparens f = rparens.
Proof.
  induction f.
  - (* Atom *) simpl. reflexivity.
  - (* Not *) simpl. rewrite. IHf. reflexivity.
  - (* And *) simpl. rewrite. IHf1. IHf2. reflexivity.
  - (* And *) simpl. rewrite. IHf1. IHf2. reflexivity.
  - (* And *) simpl. rewrite. IHf1. IHf2. reflexivity.
  - (* And *) simpl. rewrite. IHf1. IHf2. reflexivity.
Qed.

Theorem lparens_eq_rparens' : forall f : formula, lparens f = rparens f.
Proof.
  Induction f; reflexivity.
Qed.
\end{verbatim}

\section{Semantics}
\subsection{Semantics of Lp formulas}

What does \(p\) mean?
\[\mat{
    p & q\\
    0 & 0\\
    1 & 1
}
\mat{
  p & \neg p\\
  0 & 1\\
  1 & 0
}
\mat{
  p & q & p\wedge q\\
  0 & 0 & 0\\
  1 & 0 & 0\\
  0 & 1 & 0\\
  1 & 1 & 1
}
\mat{
  p & q &\neg p& \neg p\wedge q\\
  0 & 0 & 1 & 0\\
  1 & 0 & 0 & 0\\
  0 & 1 & 1 & 1\\
  1 & 1 & 0 & 0
}
\]
\[(\neg):\{0,1\}\to\{0,1\}\]
\[(\diamond):\{0,1\}\times\{0,1\}\to\{0,1\}\]

\bbox
\begin{defn}[Truth Evaluation]\label{defn:truth_evaluation}
  A \TB{truth evaluation} is a mapping from proposition symbols to
  truth values.
  \[t:\TR{Atom}\Lp\to\{0,1\}\]
\end{defn}
\ebox

\bbox
\begin{defn}
  Evaluation of formula \(A\in\TR{Form}\Lp\) under a truth evaluation
  \(t\).
  \bboxnote
  \begin{nota}[Eval Function]
    \(A^t\)
  \end{nota}
  \ebox
  
  \TB{Case 1:} \(A=p,\;p\in\TR{Atom}\Lp\). Then \(A^t=p^t=t(p)\).
  
  \TB{Case 2:} \(A=\neg B\).  Then \(A^t=(\neg B)^t\).
  Note that \(\neg(B^t)\) is wrong, because \(\neg\) is from syntax, and \(0\)
  is from semantics. So
  \[(\neg B)^t=\begin{cases}0&:B^t=1\\ 1&:B^t=0\end{cases}\]

  \TB{Case 3:} \(A=B\wedge C\).
  \[A^t=(B\wedge C)^t=\begin{cases}1&:B^t=1\T{ and }C^t=1\\ 0&:\T{otherwise}\end{cases}\]

  \TB{Case 4:} \(A=B\vee C\).
  \[A^t=(B\vee C)^t=\begin{cases}1&:B^t=1\T{ or }C^t=1\\ 0&:\T{otherwise}\end{cases}\]

  \TB{Case 5:} \(A=B\to C\).
  \[A^t=(B\to C)^t=\begin{cases}1&:B^t=0\T{ or }C^t=1\\ 0&:\T{otherwise}\end{cases}\]
  \bboxnote
  \[\mat{
      p & q & p\to q\\
      1 & 1 & 1\\
      1 & 0 & 0\\
      0 & 1 & 1\\
      0 & 0 & 1
  }\]
  \ebox

  \TB{Case 6:} \(A=B\leftrightarrow C\).
  \[A^t=(B\leftrightarrow C)^t=\begin{cases}1&:B^t=C^t\\ 0&:\T{otherwise}\end{cases}\]
\end{defn}
\ebox

\bbox
\begin{thm}
  For all \(A\in\TR{Form}\Lp[\neg,\vee,\wedge]\) and \(\forall t\), \(\Delta(A)^t
  =(\neg A)^t\) where
  \[\Delta(A):=\begin{cases}
    \neg p &:\T{if }A=p\T{ for some }p\in\TR{Atom}\Lp[\neg\vee\wedge]\\
    \neg\Delta(B)&:A=\neg B,\;B\in\TR{Form}\Lp[\neg\vee\wedge]\\
    \Delta(B)\vee\Delta(C)&:A=B\wedge C\\
    \Delta(B)\wedge\Delta(C)&:A=B\vee C
  \end{cases}\]
  \[\Delta:\TR{Form}\Lp[\neg\vee\wedge]\to\TR{Form}\Lp[\neg\vee\wedge]\]
\end{thm}
\ebox

\bboxex
\TB{Example.}
\[\Delta(\neg p\wedge q)=\neg\neg p\vee\neg q\]
\ebox

\bboxproof
\begin{proof}
  Let \(R(A)\) be the property that \(\Delta(A)^t=(\neg A)^t\). 

  \TB{Case 1:} \(A=p\).
  \[\Delta(A)^t=\Delta(p)^t=(\neg p)^t\]
  \[(\neg A)^t=(\neg p)^t\implies R(A)\T{ holds}\]

  \TB{Case 2:} \(A=\neg B\).
  \[\Delta(A)^t=\Delta(\neg B)^t=(\neg\Delta(B))^t\]
  \[\T{IH: }\Delta(B)^t=(\neg B)^t\]
  \[=\begin{cases}
    1&:\Delta(B)^t=0\\
    0&:\Delta(B)^t=1
  \end{cases}
  \]
  \[=\begin{cases}
    1&:(\neg B)^t=0\\
    0&:(\neg B)^t=1
  \end{cases}\]
  So by IH, case 2 holds.
\end{proof}
\ebox

Recap: \(t:\TR{Atom}\Lp\to\{0,1\}\).

\(A^t:=\) truth value of \(A\) under thrugh valuation \(t\)

\bbox
\begin{defn}[Satisfiable under t]\label{defn:satisfiable_under_t}
  A formula \(A\) is \TB{satisfiable} if there exists \(t\) such that 
  \(A^t=1\).
\end{defn}
\ebox

\bbox
\begin{defn}[Unsatisfiable]\label{defn:unsatisfiable}
  a formula \(A\) is unstatisfiable if for all \(t\), \(A^t=0\).
\end{defn}
\ebox

\bboxex
\begin{exam}
  \[p\is\TM{satis}\]
  \[p\wedge\neg p\is\TM{unsatis}\]
  \[p\vee \neg p\is\TM{satis}\]
  \[\T{in some sense, }p\vee\neg p=1\]
\end{exam}
\ebox

\bbox
\begin{defn}[Tautology]\label{defn:tautology}
  A formula is a \TB{tautology} if \(\forall t\), \(A^t=1\).
  \bboxex
  \begin{exam}
    \(p\vee\neg p\)
  \end{exam}
  \ebox
  \bboxnote
  \begin{nota}
    a unsatisfiable formula under \(t\) is called a \TB{contradiction}.
  \end{nota}
  \ebox
\end{defn}
\ebox


\bboxnota
\begin{nota}[Tautology]
  If \(A\) is a tautology, then \(A\mathbin{\mid\!=\!\mid}\tau\).
\end{nota}
\ebox


\bbox
\begin{defn}[Satisfiable set of Formulas]
  a set \(\Sigma\mathbin{:\subseteq}\Lp\) is called \TB{satisfiable} if
  \(\exists t\st\forall A\in\Sigma,\;A^t=1\).
\end{defn}
\ebox
A truth evaluation is basically defining the variable to true or false, then
evaluating the formula.
\bboxex
\begin{exam}
  Let \(\satis\) be the satifiable adjective.
  \[\{p\}\is \satis\]
  \[\{p,\neg p\}\is \neg \satis\]
\end{exam}
\ebox

\bbox
\begin{defn}[Unsatisfiable set of Formulas]
  A set \(\Sigma\is\neg\satis\) when \(\forall t\), 
  \[\exists A\in\Sigma,\;A^t=0\]
\end{defn}
\ebox

\bboxex
\begin{exam}
  \(\emptyset\)? It's satisfiable.
\end{exam}
\ebox

\bbox
\begin{exam}[Infinite Sigma]
  \(\Sigma:=\{p|p\in\TR{Atom}\Lp\}\). Define \(t:t(p_i)=1\).
\end{exam}
\ebox

\bbox
\begin{defn}[Arguement]
  Consists of a set of premises and a conclusion. The arguement is
  valid when the conclusion follows from the premises.

  A formula \(A\) is a tautological consequence of \(\Sigma\subseteq\TR{Form}\Lp\)
  if \(\forall t,\;\Sigma^t=1\implies A^t=1\).

  \[
    \Sigma^t:=\begin{cases}1&\forall A\in\Sigma,\;A^t=1\\ 0&\TR{otherwise}\end{cases}
  \]
  \[
    \Sigma^t=\bigwedge_{A\in\Sigma}A^t
  \]

  This is saying \(\Sigma\implies A\)
  
\end{defn}
\ebox

\bboxex
\begin{exam}
  \(\Sigma=\{p\to q, p\}\) \(A=q\). So \(\Sigma\implies A\). 
  Hypothetical syllagism.
\end{exam}
\ebox

\bbox
\begin{nota}
  \(\Sigma\models A\), means \(A\) is a logical consequence of \(\Sigma\).
\end{nota}
\ebox

\bboxex
\begin{exam}
  \(\{p\}\not\models\neg p\)
\end{exam}
\ebox

\bbox
\begin{defn}[Maximally Satisfiable]
  a set \(\Sigma\subseteq\TR{Form}\Lp\) is maximally satisfiable
  if \(\forall A\in\TR{Form}\Lp,\;\Sigma\models A\T{ xor }\Sigma\models\neg A\).

  Note that \(\Sigma\models\neg A\not\equiv\Sigma\not\models A\).
\end{defn}
\ebox

\bboxex
\begin{exam}[The Infinite Atomic Set is Maximally Satisfiable]
  \(\{p\}\models p,\;\{p\}\not\models\neg p,\;
  \{p\}\not\models q,\;\{p\}\not\models\neg q\). So \(\{p\}\) isn't
  maximally satisfiable.
  \[\{p_1,p_2,\dots\}:=\Sigma\]
  \[t(p_i):=1\implies\Sigma^t=1\]
  \[\implies\Sigma\models A\T{ xor }\Sigma\models\neg A\]
\end{exam}
\ebox

\bbox
\begin{defn}[Uniquely Satisfiable]
  \(\Sigma\) is uniquely satisfiable if
  \[
    \exists!t\st\Sigma^t=1
  \]
\end{defn}
\ebox


\bbox
\begin{thm}[Uniquely Satisfiable iff Maximally Satisfiable]
  Suppose \(\Sigma\) is satisfiable. Then \(\Sigma\) is 
  uniquely satisfiable \TB{iff} \(\Sigma\) is maximally satisfiable.
\end{thm}
\ebox

\bboxproof
\begin{proof}
  \((\implies)\). Assume \(\Sigma\) maximimally satisfiable. 
  Assume by contradiction that there is \(t_1,t_2\st\Sigma^{t_1}=1\) and
  \(\Sigma^{t_2}=1\), so \(\Sigma\) isn't uniquely satisfiable.

  So \(t_1\neq t_2\implies\exists p\st t_1(p)\neq t_2\).

  \(t_1(p)=1\iff t_2(p)=0\). Let \(t_1(p)=1\) and \(t_2(p)=0\).

  We know \(\Sigma\models p\mathbin{\TR{xor}}\Sigma\models\neg p\).

  \TB{Case 1:} \(\Sigma\models p\). \(p^{t_1}=1\T{ and }p^{t_2}=1\) which
  is a contradiction.

  \TB{Case 2:} \(\Sigma\models\neg p\). \((\neg p)^{t_1}=1\T{ and }(\neg p)^{t_2}=1\)
  which is a contradiction.
\end{proof}
\ebox

% lecture 5

% recap: syntax, then semantics
% truth evaluation
% t: Atom(Lp)->{0,1}
%
% recap: syntax, then semantics
% truth evaluation
% t: Atom(Lp)->{0,1}
% A^t is A under t. (sets the atoms to 1,0; evaluate it)
% 
% E |= A defn: E |= A iff \forall t, E^t = 1 ==> A^1 == 1

% E is satis iff
%   \exists t \st E^t = 1


\bboxexam
\begin{exam}
  If \(\Sigma\is\neg\satis\implies\Sigma\models A\) holds when?
  \(\forall t,\;\Sigma^t=0\). So \(\Sigma=1\) is always false.
  So that always implies \(A^t=1\). So it always holds. YAY.
\end{exam}
\ebox


\bboxdefn
\begin{defn}[Models]
  \(\Sigma\models A\TB{ iff }\Sigma^1=1\implies A^t=1\).
\end{defn}
\ebox

\bboxexam
\begin{exam}
  \(A\is\neg\satis\implies\Sigma\implies A\) only sometimes
\end{exam}
\ebox

\bboxexam
\begin{exam}
  \(A\is\tau\implies\Sigma\models A\) always.
\end{exam}
\ebox

\bboxexam
\begin{exam}
  \(\Sigma\models A\implies \Sigma\cup\{\neg A\}\) is never satis.
  \[\Sigma^t=1\implies A^t=1\]
  \[(\Sigma\cup\{\neg A\})^t=1?\]
  \[\Sigma^t=1\implies\neg A=0\]
  \[\Sigma^t=0\implies\T{ already not satis}\]
\end{exam}
\ebox


\bboxexam
\begin{exam}
  \(\Sigma\is\satis,\;\Sigma'\subseteq\Sigma\implies\Sigma'\is\satis\)
  If that \(t\) satis sigma, then it also satis sigma'.
\end{exam}
\ebox

\bboxexam
\begin{exam}
  \(\Sigma\is\satis,\;\Sigma'\supseteq\Sigma\implies\Sigma'\is\T{sometimes }\satis\)
\end{exam}
\ebox

\newcommand{\logeq}{\mathbin{\mid\!=\!\mid}}

\bboxdefn
\begin{defn}[Logical Equivalence]
  \(A\) is logically equivalent with \(B\) if
  \[A\models B\T{ and }B\models A\]
  \[A\logeq B\]
\end{defn}
\ebox

\bboxproof
\begin{proof}[Logical Equivalence is a Equivalence Relation Proof]
  Lol nevermind.  
\end{proof}
\ebox

\bboxexam
\begin{exam}[Important Logical Equivalences]\(\,\)
  \begin{itemize}
    \item \TB{Conj is Abelian:} \(A\wedge B\logeq B\wedge A\)
    \item \TB{Disj is Abelian:} \(A\vee B\logeq B\vee A\)
    \item \TB{Equi is Abelian:} \(A\leftrightarrow B\logeq B\leftrightarrow A\)
    \item \(\diamond\)\TB{ is Assoc:} \(A\diamond(B\diamond C)\logeq(A\diamond B)\diamond C\)
      for \((\diamond)\in\{\wedge,\vee,\leftrightarrow\}\)
    \item \TB{Dist of Disj:} \((A\wedge B)\vee C\logeq(A\vee C)\wedge(B\vee C)\)
    \item \TB{Dist of Conj:} \((A\wedge B)\wedge C\logeq(A\wedge C)\vee(B\wedge C)\)
    \item \TB{De Morgan's Disj:} \(\neg(A\vee B)\logeq\neg A\wedge\neg B\)
    \item \TB{De Morgan's Conj:} \(\neg(A\wedge B)\logeq\neg A\vee\neg B\)
    \item \(\Delta(A)\logeq\neg A\)
    \item \(A\logeq\neg\neg A\)
    \item \(A\vee\neg A\logeq\tau\)
    \item \(A\wedge\neg A\logeq\mathcal C\).
    \item \TB{Iden of Disj:} \(A\vee\mathcal C\logeq A\)
    \item \TB{Iden of Conj:} \(A\wedge\tau\logeq A\)
    \item \TB{Domination of Conj:} \(A\wedge\mathcal C\logeq\mathcal C\)
    \item \TB{Domination of Disj} \(A\vee\tau\logeq \tau\)
    \item \(A\rightarrow B\logeq\neg A\vee B\)
    \item \TB{Contrapositive Equiv:} \(A\rightarrow B\logeq \neg B\rightarrow\neg A\)
    \item \(A\leftrightarrow B\logeq(A\rightarrow B)\wedge(B\rightarrow A)\)
    \item \TB{Absorption of Disj:} \((A\wedge B)\vee A\logeq A\)
    \item \TB{Absorption of Conj:} \((A\vee B)\wedge A\logeq A\)
    \item \((A\wedge B)\vee(A\wedge\neg B)\logeq A\)
    \item \((A\vee B)\wedge(A\vee\neg B)\logeq A\)
  \end{itemize}
\end{exam}
\ebox

\bboxthm
\begin{thm}[Replaceabiliy]
  If \(B\logeq C\), then replacing some occurances of B inside A
  gives a logically equivalent formula. We can do beta reduction.
\end{thm}
\ebox

\bboxexam
\begin{exam}[Pierce's Law]
  \(((A\rightarrow B)\rightarrow A)\rightarrow A\)
  \begin{align*}
    &\logeq((\neg A\vee B)\to A)\to A\\
    &\logeq(\neg(\neg A\vee B)\vee A)\to A\\
    &\logeq((A\wedge\neg B)\vee A)\to A\\
    &\logeq A\to A\\
    &\logeq\neg A\vee A\\
    &\logeq A\vee\neg A\\
    &\logeq 1_\tau
  \end{align*}
\end{exam}
\ebox

\subsection{Normal Forms}

\bboxdefn
\begin{defn}[Literals]
  a formula is a \TB{literal} if it is a either \(p\) or \(\neg p\) for some \(p\in\TR{Form}\Lp\). 
\end{defn}
\ebox

\bboxdefn
\begin{defn}[Conjuctive Clause]
  a formula is a \TB{conjuective clause} is a conjunction of literals.
\end{defn}
\ebox

\bboxdefn
\begin{defn}[Disjunction Clause]
  a formula is a \TB{disjunction clause} is a disjunction of literals.
  \bboxexam
  \begin{exam}
    \(p\) is a disjunctive clause with 1 disjoint.
  \end{exam}
  \ebox
\end{defn}
\ebox

\bboxdefn
\begin{defn}[Conjunctive Normal Form (CNF)]
  A formula is a (CNF) if it is a conjunction of disj clauses.
\end{defn}
\ebox

\bboxdefn
\begin{defn}[Disjunctive Normal Form (DNF)]
  A formula is a (DNF) if it is a disjunction of conj clauses.
  \bboxexam
  \begin{exam}
    \((p\wedge\neg q)\vee\neg p\), 
    \(\neg p\vee p\vee q\)
  \end{exam}
  \ebox

\end{defn}
\ebox

\(\neg p\) is a literal, a disj clause, a conj clause, in CNF, in DNF.

\bboxexam
\begin{exam}[into DNF]
  Transform \(p\leftrightarrow\neg q\) into (DNF).
  \begin{enumerate}
    \item Use a truth table. \((p\wedge\neg q)\vee(\neg p\wedge q)\).
      \(\mat{p&q&p\leftrightarrow\neg q\\ 1&1&0\\ 1&0&1\\ 0&1&1\\ 0&0&0}\).
  \end{enumerate}
  Transform \(p\leftrightarrow\neg q\) into (CNF).
  \begin{enumerate}
    \item Use a truth table of \(\neg(p\leftrightarrow\neg p)\). 
      \(\logeq(p\wedge q)\vee(\neg p\wedge \neg q)\).
      \(\overset\neg\longrightarrow(\neg p\vee \neg q)\wedge(p\vee q)\).
  \end{enumerate}
\end{exam}
\ebox

\bboxexam
\begin{exam}[into DNF]
  Transform \(p\leftrightarrow\neg q\) into (DNF).
  \begin{enumerate}
    \item Use a truth table. \((p\wedge\neg q)\vee(\neg p\wedge q)\).
      \(\mat{p&q&p\leftrightarrow\neg q\\ 1&1&0\\ 1&0&1\\ 0&1&1\\ 0&0&0}\).
    \item \(p\iff\neg q\logeq(p\implies\neg q)\wedge(\neg q\implies p)\logeq
      (\neg p\vee q)\wedge(q\vee p)\T{ is CNF}\logeq \)
  \end{enumerate}
\end{exam}
\ebox

\subsection{Essential Laws of Propositional Calculus}

We can do algebriac simplification over these formulas. I.e. boolean algebra.


\bboxdefn
\begin{defn}[Boolean Algebra]\ 
    \begin{enumerate}
        \item underlying set \(B\)
        \item binary operations \(\{(+),(\cdot)\}\)
        \item unary operator \(\{\overline{\ \cdot\ }\}\) "complement"
        \item nullary operator \(1:B\ 0:B\). The functions that take no input and give 1
        \item Laws:
            \begin{enumerate}
                \item \(x+0=x\)
                \item \(x\cdot 1=x\)
                \item \(x+\overline x=1\)
                \item \(x\cdot\overline x=0\)
                \item \((x+y)+z=x+(y+z)\)
                \item \((x*y)* z=x*(y*z)\)
                \item \(x+y=y+x\)
                \item \(x*y=y*x\)
            \end{enumerate}
    \end{enumerate}
\end{defn}
\ebox


\bboxexam
\begin{exam}
    Any set \(S\) generates a boolean algebra. 
    \begin{enumerate}
        \item \(B=\PP(S)\)
        \item \(+=\cup\)
        \item \(\cdot=\cap\)
        \item \(\overline{A}=S\backslash A\)
        \item \(0=\emptyset\)
        \item \(1=S\)
        \item Laws:
            \begin{enumerate}
                \item \(A\cup\emptyset=A\)
                \item \(A\cap S=A\)
                \item \(A\cup(S\backslash A)=S\)
                \item \(A\cap(S\backslash A)=\emptyset\)
            \end{enumerate}
    \end{enumerate}
\end{exam}
\ebox

\bboxexam
\begin{exam}
    \(\TR{Form}\Lp\) generate a boolean algebra. \(B:=\TR{Form}\Lp\)
    \begin{enumerate}
        \item \(+=\vee\)
        \item \(\cdot=\vee\)
        \item \(\overline A=\neg A\)
        \item \(0=\T{any contradiction}\)
        \item \(1=\T{any tautology}\)
        \item Laws:
            \begin{enumerate}
                \item \(A\lor 0\logeq A\)
                \item \(A\land 1\logeq A\)
                \item \(A\lor\neg A\logeq 1\)
                \item \(A\land\neg A\logeq 0\)
                \item ...
            \end{enumerate}
    \end{enumerate}
\end{exam}
\ebox

\bboxexam
\begin{exam}
    Disjunctive Normal Form.
    \[
        xyz+x\overline yz+xy\overline z
    \]
    \[
        \T{2 OR GATES, 2 NOT GATES, 6 AND GATES}
    \]
    \[
        \logeq x(yz+\overline yz+y\overline z)
    \]
    \[
        \logeq x(yz+yz+\overline yz+y\overline z)
        \T{ since }b+b=b
    \]
    \[
        \logeq x((y+\overline y)z+y(z+\overline z))
    \]
    \[
        \logeq x(y+z)
    \]
\end{exam}
\ebox

\subsection{Amount of Connectives per Formula}


\bboxthm
\begin{thm}[Formula to CNF]
    Any Formula in \(\Lp\) is logically eq to atleast one forumla in CNF (also DNF).
\end{thm}
\ebox

\bboxnote
\begin{note}
    We remove \(\implies\) and \(\iff\) using laws.

    Push \(\neg\) inside \(\land/\lor\) using de Morgan's laws

    So we only need \(\{\neg,\land,\lor\}\)
\end{note}
\ebox

How many connectives are there? A connective is a n-ary boolean function.

\begin{enumerate}
    \item \(n=1:4\T{ connectives}\)
    \item \(n=2:16\T{ connectives}\)
\end{enumerate}

We have \(2^n\) possible inputs, considering a binary string as input. For each input,
there are \(2\) choices. So we have \(2^{2^n}\) choices.



\bboxdefn
\begin{defn}
    A set of logical connectives \(S\) is adequate if the connectives in \(S\) 
    are capable of expressing any set of n-ary connectives.
\end{defn}
\ebox

\bboxdefn
\begin{defn}[Adequate Set of Connectives]
    \(S\) connectives is adequate if for any n-ary boolean function \(f:\{0,1\}^n\to\{0,1\}\)
    and for any set of prop symbols \(p_1,p_2,\dots,p_n\) there is a 
    \(A\in\TR{From}\Lp\) using only connectives in \(S\) and prop symbols of
    \(p_1,\dots,p_n,\;\st\)
    \[
        f(p_1,\dots,p_n)\logeq A
    \]
\end{defn}
\ebox

\bboxthm
\begin{thm}[Std Connectives are Adequate]
    \(\{\neg,\lor,\land\}\) is ade.
\end{thm}
\ebox


\bboxthm
\begin{thm}[And and Neg are Ade]
    \(\{\neg,\land\}\) is ade.
\end{thm}
\ebox

\bboxproof
\begin{proof}
    We already know that
    \[
        \{\neg,\lor,\land\}\is\TM{ade}
    \]
    \[
        \T{Show }\forall c\in\{\neg,\land,\lor\}\T{ can be expressed in }
        \{\neg,\land\}
    \]
    \[
        \neg p\logeq\neg p
    \]
    \[
        p\land q\logeq p\land q
    \]
    \[
        p\lor q\logeq\neg(\neg p\land\neg q)
    \]
\end{proof}
\ebox


\bboxthm
\begin{thm}
    \[\{\neg,\lor\}\]
    \[
        \{\neg,\implies\}
    \]
\end{thm}
\ebox


\bboxnote
\begin{thm}[NOR is adequate]
    The non-disjunction, NOR is adequate alone.
    \[
        \mat{p&q&\TR{NOR}(p,q)\\ 1&1&0\\ 1&0&0\\ 0&1&0\\ 0&0&1}
    \]
\end{thm}
\ebox

\bboxproof
\begin{proof}
    We know \(\{\neg,\lor\}\) is ade.
    \[
        \neg p\logeq\TR{NOR}(p,p)
    \]
    \[
        p\lor q\logeq\neg\neg(p\lor q)\logeq\neg\TR{NOR}(p,q)
        \logeq\TR{NOR}(\TR{NOR}(p,q),\TR{NOR}(p,q))
    \]
\end{proof}
\ebox

\bboxthm
\begin{thm}[NAND is ade]
    \(\{\TR{NAND}\}\) is ade.
\end{thm}
\ebox

\bboxproof
\begin{proof}
    EX.
\end{proof}
\ebox

\bboxthm
\begin{thm}[And isn't Ade]
    \(\{\wedge\}\) isn't ade.
\end{thm}
\ebox

\bboxproof
\begin{proof}
    Assume \(\{\wedge\}\) is ade, for the sake of contradiction. 
    So that means every formula can be written in terms of only \(\land\).
    So \(\neg p\logeq p_1\land p_2\land p_3\land\cdots\logeq A\), where
    \(p_i=f_i(p)\) for some \(f_i\). Build out of \(p\).

    Let \(t\) be a truth evaluation s/t \((\neg p)^t=1\).
    \[
        \implies p^t=0
    \]
    \[
        \T{We know }\neg p\logeq A
    \]
    \[
        \implies A^t\logeq 1
    \]
    \[
        \T{claim }A^t=0
    \]
    This is a contradiction.

    \TB{Claim: }For any \(A\in\TR{Form}\Lp[\wedge]\), if \(p^t=0\) then \(A^t=0\).

    \bboxproof
    \begin{proof}
        Induction on \(A\).
        \TB{Base Case: }\(A=p\).
        \[
            A^t=p^t=0
        \]
        \TB{Inductive Case: }\(A=B\land C,\T{ for some }B,C\in\TR{Form}\Lp[\wedge]\)
        \[
            \T{IH: }B^t=C^t=0
        \]
        \[
            A^t=(B\land C)^t=0
        \]
        So by induction, \(A^t=0\). So the claim holds, so the proof follows.
    \end{proof}
    \ebox

\end{proof}
\ebox


% <++>
\end{document}


























% scrolloff buffer
