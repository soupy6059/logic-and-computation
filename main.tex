\documentclass[12pt]{article}
\usepackage{my_preamble}

% Document Starts
\begin{document}

\maketitle 

\begin{abstract}
\end{abstract}

\tableofcontents

\pagebreak
% <++>


\section{Proposition}

An atomic prop cannot be broek down into smaller propositions.

A compound proposition is composed of atomics props.

\TB{Atomic}
\begin{itemize}
  \item I am graduating.
  \item I am applying for grad school.
\end{itemize}

\TB{Compound}
\begin{itemize}
  \item I am not graduating
  \item I am graduating implies im applying for grad school
\end{itemize}

\section{Logical Arguments}

An \TB{argument} is a set of props, consiting of zero or more premises.

\bboxex
\TB{Premises:} If I am applying for grad schools, then I must be graduating.
I am graduating.
\ebox
\bboxex
\TB{Conclusion:} I am applying for grad school.
\ebox

If the concl doesn't follow from prem then the argument is invalid.

% lecture 2


\section{Propositional Logic}
\subsection{Notations and Lp}
\newcommand{\Lp}{\TM{Lp}}

\bbox
\begin{nota}[Symbols]\label{nota:sym}
  \(\,\)\begin{itemize}
    \item \TB{Proposition Symbols.} Used for atomic formulas. We'll use lowercase 
      letters, \(\{a,b,c,\dots\}\).
    \item \TB{Connections.} \(\neg,\wedge,\vee,\to,\leftrightarrow\).
    \item \TB{Parems.} Denotes order.
  \end{itemize}
  Let Lp be the language of propositional logic.
  \[\wedge\wedge\vee\is\TM{not\;legal}\]
  \[(p\is\TM{not\;legal}\]
  We defined a tokenizer. We'll now define the \TB{parser}.
\end{nota}
\ebox


\bbox
\begin{defn}[Well Formed Expressions (WFE)]\label{defn:well_exp}
  \(\,\)
  \begin{enumerate}
    \item a propositional symbol is a well formed expression.
      \[p\is\TM{WFE}\]
    \item If \(A\in\mathrm{Form}\TM{Lp}\implies(\neg A)\in\mathrm{Form}\TM{Lp}\).
    \item If \(A,B\in\mathrm{Form}\Lp\implies (A\wedge B)\in\mathrm{Form}\Lp\).
    \item If \(A,B\in\mathrm{Form}\Lp\implies (A\vee B)\in\mathrm{Form}\Lp\).
    \item If \(A,B\in\mathrm{Form}\Lp\implies (A\to B)\in\mathrm{Form}\Lp\).
    \item If \(A,B\in\mathrm{Form}\Lp\implies (A\leftrightarrow B)\in\mathrm{Form}\Lp\).
  \end{enumerate}
\end{defn}
\ebox
\begin{verbatim}
data WFE t = PropSym t |
             ExprBin (t -> t -> t) (WFE t) (WFE t) |
             ExprUn (t -> t) (WFE t)

eval :: WFE t -> t
eval wfe = case wfe of
  PropSym t -> t
  ExprBim fn l r -> fn (eval l) (eval r)
  ExprUn fn u -> fn (eval u)


neg :: Bool -> Bool
neg pred = if pred then False else True

conj :: Bool -> Bool -> Bool
conj a b = if a then b else False

inj :: Bool -> Bool -> Bool
inj a b = (neg a) `conj` (neg b)
     -- = (conj `on` neg) a b

xor :: Bool -> Bool -> Bool
xor a b = (a `ing` b) `conj` ((neg . conj) a b)
\end{verbatim}
To make inline Lp work, we need to establish operator prior and associativity.
\bboxnote
\begin{nota}[Operator Priority]\label{nota:oper_prior} Operator prior is as follows:
\begin{enumerate}
  \item \(\neg\)
  \item \(\wedge\is\TM{left\;assoc}\)
  \item \(\vee\is\TM{left\;assoc}\)
  \item \(\implies\is\TM{right\;assoc}\)
  \item \(\iff\is\TM{left\;assoc}\)
\end{enumerate}
\end{nota}
\ebox
\bboxex
\[((\neg p)\vee q)=\neg p\vee q\]
\[(p\wedge q)\vee(r\wedge p)=p\wedge q\vee r\wedge q\]
\[p\to q\to r=p\to(q\to r)\]
\[p\wedge q\wedge r=(p\wedge q)\wedge r\]
\ebox

\section{Translating English into Prop Logic}

\subsection{Examples}
\bboxex
\begin{equation*}
  \begin{split}
    s&:=\T{I am applying to grad schools}\\
    j&:=\T{I am applying to jobs}\\
    g&:=\T{I am graduating}\\
    \T{s or j}&=s\vee j\\
    \T{i am either S or J but not S and J}&=(s\vee j)\wedge\neg(s\wedge j)\\
                                          &=s\iff\neg j\\
                                          &=(s\implies\neg j)\wedge(\neg j\implies s)\\
    (a\vee b)\wedge\neg(a\wedge b)&:=a\oplus b\\
    \T{s if g}&=g\implies s\\
    \T{s only if g}&=s\implies g\\
    \T{storm}\implies\T{rain}&=\T{rain if storm}\\
                             &=\T{it's raining if it's storming}\\
                             &=\T{storm only if rain}\\
                             &=\T{it's storming only if it's raining}\\
    \T{g is sufficient for s}&=g\implies s\\
    \T{g is necessary for s}&=s\implies g\\
    \T{Although g, i am not j}&=g\wedge\neg j\\
    \oplus=\neg\circ\leftrightarrow
  \end{split}
\end{equation*}
\ebox


\bbox
\begin{lem}[Balanced Paranthesis]\label{lem:balance_parans_in_lp}
  Every formula in \(\mathrm{Form}\Lp\) has balanced paranths.
\end{lem}
\ebox
\bboxproof
\begin{proof}
  Let \(A\) be an arbitrary formula in \(\mathrm{Form}\Lp\). The following
  proof is by \TB{structural induction}.

  Let \(LP(A)\) be the number of Left paranthesis' in \(A\). Let \(RP(A)\) be 
  the number of Right paranthesis' in \(A\).

  \TB{Base Case}: \(A\) is atomic, \(A=p\) for some prop p.
  \[LP(A)=RP(A)=0\]

  \TB{Inductive Case 1}: \(A=\neg B\) for some \(B\in\mathrm{Form}\Lp\).
  Our IH says \(LP(B)=RP(B)\).
  \[LP(A)=LP((\neg B))=1 + LP(B)=1+RP(B)=RP((\neg B))=RP(A)\]
\end{proof}
\ebox

% end of lec 2 

% ADD hyperref
% <++>
\end{document}


























% scrolloff buffer
